\chapter{Introduction}

One of the most popular observations in computer science is Moore's Law.
It was introduced in the mid-1960s when Gordon E. Moore described an empirical relationship \cite{Moore}
projecting that the number of transistors per IC (Integrated Circuit) will increase twofold every year.
He predicted that this trend will continue for at least a decade, but remarkably enough this is still true 50 years later. 
For most of that time CPU (Central Processing Unit) consisted of only one single processing core with performance growing through a combined increase of clock frequency speed and number of components per IC. At some point though, CPUs ran into a bottleneck,
further clock frequency increases would come with drastically higher heat production and energy consumption. Thus it became clear that to develop CPUs which will meet ever increasing demand for computing power, new architectural designs have to be engineered.
\newline
Before multi-core CPUs became commercialy available, Intel came up with a new idea of boosting performance, SMT (Simultaneous Multi-Threading) and it's implementation - Hyper-Threading. It was first included with Pentium 4 and Xeon processors in 2002. Architecturally, a processor with Hyper-Threading Technology consists of two logical processors per core, each of which has its own processor architectural state. Each logical processor can be individually halted, interrupted or directed to execute a specified thread, independently from the other logical processor sharing the same physical core.
Unlike a traditional dual-processor configuration that uses two separate physical processors, the logical processors in a hyper-threaded core share the execution resources. Intel claimed that they would get a performance boost around 15 to 30 \% [2] compared to other non-Hyper-
Threaded CPUs and only increase the size of the die with 5\%. AMD in response developed the Bulldozer architecture (released 2011), based on CMT (Cluster Multi-Threading), but it never delivered satisfactory performance increases, thus it was replaced by the proper SMT Zen architecture in 2017, which will be described later.
\newline

Thus, in 2001, worlds first multi-core processor (IBM's POWER4 ) was introduced. Multi-core architecture allowed designers to further increase performance while keeping the cost down.
But with the introduction of multi-core the developers have had to change
their mindset when writing program code. Not only would they have to
divide the program in such a way that each part could run concurrently,
they also had to think about synchronization when more than one core have
access to shared data. Taking advantage of multi-core to get the desired
performance increase, these concurrent parts will have to run in parallel
and all necessary sequential fraction of the program needs to be kept to a
minimum (Amdahl�s law).
While a single core CPU only could execute a single sequence of instructions, multi-core could execute many sequences at once
Before they were introduced, Intel came up with another idea for boosting performance. With the introduction of Intel Pentium 4 processor in 2002 came the technology of Hyper-Threading. 


\section{Goals and scope}

\section{Thesis structure}
